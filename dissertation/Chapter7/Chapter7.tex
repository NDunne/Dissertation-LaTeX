% !TEX root =  ../report.tex
% !TeX spellcheck = en-GB

\section{Conclusion}
\label{s:conc}

This project was developed as an investigation into a new optimisation for the GPU code generation of the OP2 framework. As part of this investigation, a fully functioning implementation of the technique was designed and completed, which can be applied to other OP2 applications that utilise the C API. 
\par
The implementation successfully augments the existing code generation with the ability to execute JIT compiled code, and applies an optimisation that is made based on the inputs of the program: defining the constant values from the input as pre-processor literals. This could only be done at run-time, and would not be possible by the existing OP2 generated code.
\par
The results from benchmarking were that there was only a small speedup to the run-time, but it would appear that the speedup is linear and the cost is constant with respect to problem size. It is important to draw the distinction that it was the run-time optimisation of defining of constants which was only providing limited speedup; and that these results do prove that there is benefit to be gained from applying JIT compilation in this context.
\par
It is likely that a further optimisation such as loop blocking will be implemented on top of this implementation in the future, to improve on the per-iteration speed-up. Adding additional run-time optimisations to OP2 will be easier as a result of the work completed for this project, as they will be extensions to this implementation .
\par
Overall, the project was a successful investigation, which has provided a useful contribution to an open source library. The results gathered will inform and benefit future contributions, and the implementation completed will become part of a framework which provides benefit to many industrial HPC applications.
