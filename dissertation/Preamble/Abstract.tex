% !TEX root =  ../report.tex

%
\vspace*{\fill}
\begin{adjustwidth}{55pt}{55pt}

\section*{Abstract}
\addcontentsline{toc}{section}{Abstract}
The OP2 Domain Specific Language was created to simplify the process of writing unstructured mesh solver applications for High Performance Computing. This report details the implementation of a new optimisation to the code generation portion of the OP2 Framework, which re-compiles at run-time when the inputs to the program are known, as well as the benchmarking of this optimisation on a representative example application. For this implementation the only assertion made at run-time was to define input values declared constant as pre-processor literals in the re-compiled code, however further run-time optimisation are also discussed. The finished JIT compilation platform will aid in adding additional optimisations in the future, which could provide speed-up where defining constants could not.

\section*{Key Words}
\addcontentsline{toc}{section}{Key Words}
High Performance Computing, Unstructured Mesh, Just-In-Time Compilation
\end{adjustwidth}
\vspace*{\fill}
