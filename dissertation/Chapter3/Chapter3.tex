% !TEX root =  ../report.tex
% !TeX spellcheck = en-GB

\section{Specification}
\label{s:spec}

In order to clearly explain the design of the extension that will be made to the OP2 framework, it is important to first understand the pre-existing work-flow. The following is a high level overview of the components of OP2 that pre-date this project, so that when the new system is described in Section \ref{ss:reqs} it is easier to understand.

\subsection{Pre-Existing System}
\begin{figure}[h!]
\makebox[\textwidth][c]{
 \resizebox{1.1\textwidth}{!}{
  \begin{tikzpicture}[node distance=4cm, auto]
    \node [file] (source) {Source Files};
    \node [block, below of=source] (op2py) {CodeGen script};
    \node [file, right of=op2py, text width=6em] (sourceOp) {Modified Application File};
    \node [file, above of=sourceOp] (Kernels) {Kernels};
    \node [wblock, right of=sourceOp] (aot) {AOT Compiler};
    \node [wfile, below=1cm of aot] (op2lib) {OP2 Library Files};

    \node [rbin, right of=aot] (binary) {Program Binary};
    \node [rfile, above of=binary] (input) {Input Map};
    \node [rblock, right of=binary] (result) {Program Result};

    \tikzset{dotted box1/.style={draw=black!100, dash pattern=on 4pt off 4pt,
      inner sep=8mm, rectangle, rounded corners, line width=2pt}};

    \node (run-time) [dotted box1, fit = (input) (result), color=green!50!black] {};
    \node at (run-time.north east) [above left=2mm] (rtbox) {\textbf{Run-time}};

    \node (code-gen) [dotted box1, fit = (source) (sourceOp), color=blue!50!black] {};
    \node at (code-gen.north west) [above right=2mm] (rtbox) {\textbf{Code Generation}};

    \path [line, color=blue!50!black] (source) -- (op2py);
    \path [line, color=blue!50!black] (op2py) -- (sourceOp);
    \path [line, color=blue!50!black] (op2py) -- (Kernels);
    \path [line, color=blue!50!black] (sourceOp) -- (aot);
    \path [line, color=blue!50!black] (Kernels) -- (aot);
    \path [line, color=black] (op2lib) -- (aot);
    \path [line, color=green!50!black] (aot) -- (binary);
    \path [line, color=green!50!black] (input) -- (binary);
    \path [line, color=green!50!black] (binary) -- (result);

  \end{tikzpicture}
  }}
  \caption{Pre-existing OP2 System Diagram}
  \label{fig:aot_sys}
\end{figure}

\noindent The pre-existing OP2 workflow is shown in Figure \ref{fig:aot_sys}. The diagram starts in the top left with the Code Generation stage, beginning at the system's input: a set of Source Files. The set of Source Files cannot be empty, and must contain a Master Application file. The Master Application file is a normal C program, which makes OP2 API calls to define of sets, maps, and constants, as well as for initialising and cleaning up the OP2 execution with \verb|op_init()| and \verb|op_exit()|. It describes the application structure, and when parallel loops should be executed.
\par
The input Source Files can optionally contain additional C source and header files, included in the normal way with \verb|#include| statements, to assist with the organisation of a complex application with a large code-base. An example usage of these files would be a header file for each parallel loop, containing the function to be executed as the loop body.
\par
These Source Files are parsed by the code generation Python script, from which the output is a modified version of the master application file, and a kernel file for each parallel loop. The output files are compiled, and linked against the OP2 library files for the GPU platform to produce an executable Program Binary. The expectation is that this binary will run without error on the target hardware, taking a map as an input to produce the result desired by the application programmer.
\par
The pre-existing system is able to generate optimised code for GPUs from the high level application code, and apply optimisations ahead of time. It is not, however, able to optimise based on the inputs at all, as they are only known at run-time, after the compiler has completed and the binary is being executed. This is where the implementation for this project comes in: to provide the ability to use the input data when optimising.

\subsubsection{op2.py}
\label{ss:impl_op2}
The code generation is done using Python scripts, with the main script being \verb|op2.py|, which parses the input files to gather data, and provides this data to a number of other scripts. Each of these other scripts performs the code generation for a specific hardware platform. \verb|op2.py| uses the Python Regular Expressions (RegEx) library: \verb|re| \cite{re} to identify OP2 API calls in the Application File, and ensures certain conditions are met - for example that \verb|op_init| and \verb|op_exit| are both called at least once to initialise and clean up the OP2 execution environment.
\par
Information is also gathered during the parse about each parallel loop, including the number of the parameters and their types, and the details of the indirect data set if the loop is indirect. This stage includes some error checking, by ensuring types and dimensions are consistent throughout the application.
\par
Once the Application has been analysed, \verb|op2.py| produces a modified copy of the Application File, named \verb|[application]_op.cpp|, which is largely the same as the file provided by the application programmer, but with the addition of \verb|extern| declarations for the function each parallel loop will call: \verb|op_par_loop_[name]|. Defining a function \verb|extern| means it has external linkage \cite{linkage}, and therefore the definition of the function may be found during the linking stage of compilation, not in the current pass.
\par
The generator scripts for each platform will receive the list of loop details gathered using RegEx as its parameters, then will generate a definition of every parallel loop's execute function in the form of Kernel files. At compile-time the definitions given in these Kernel files will be linked to the extern declaration in the Modified Application File by the linker.
\par
The requirements for the code generation script that will be created as part of this project, that will produce kernels containing CUDA code with JIT compilation, will be discussed in the following section.

\subsection{New System Requirements}
\label{ss:reqs}

Implementing the new system will require work in two main areas: a new Python code generation script, and some modifications to the OP2 library itself. The OP2 Library is currently implemented in both C and Fortran, but only the C library will be modified, due to developer familiarity with the C language. OP2 does also include code generation using MatLab, however the Python script is preferable for new developments, since Python is now ubiquitous, and provides very convenient string manipulation capabilities. The following are the necessary requirements to consider this project a success.

\subsubsection{Python Script}
The new code generation script will be named \verb|op2_gen_cuda_jit.py|, and will need to perform a somewhat similar source-to-source translation process to pre-existing CUDA script for Ahead-Of-Time (AOT) compiled code. The extension required is the ability to also generate a second, altered code-base that will be compiled at run-time, as well as the original code that is compiled prior to running the executable.
\par
All code generated by the new code generation script must form valid C files, and compile using a the Intel C compiler \cite{icc} without errors. Since the project will involve generation of NVidia CUDA as well, the generated CUDA code must also be valid, and compile with the NVidia C Compiler (\verb|nvcc|) from the NVidia CUDA Toolkit \cite{nvcc,toolkit} without errors.
\par
When the resulting executable which has been compiled from the generated code is executed, it will need to invoke a re-compilation stage while it is running, and execute code that has been compiled during its runtime as part of its execution. It must produce an output within some tolerance of the expected result, obtained from executing the parallel loop iterations sequentially in an arbitrary order. The order is not significant, as OP2 enforces a restriction that the order in which elements are processed must not affect the final result, within the limits of finite-precision floating-point arithmetic \cite[p3]{op2main}.
\par
Lastly, the above requirements must be met for both Array-of-Structs and Struct\\-of-Arrays data layouts, especially when automatic SoA conversion is enabled, as this alters the generated code.

\subsubsection{Run-time Assertions}
The application's input will be a large number of data points forming an unstructured mesh which it will operate over. The optimisation that will be made for this project is "Constant Definition", built on the assertion that values declared as OP2 constants are certainly not going to change during the course of execution. To apply this optimisation, constant values provided as part of the input must be turned into \verb|#define| directives for the C pre-processor in the recompiled code. This will result in all references to the variable's identifier in the code being transformed so they are seen as a literal value rather than a memory read by the compiler.
\par
An example of how this is normally used can be seen in the CUDA example program from before (Appendix \ref{app:cudaEx}), where the size of the arrays to be added together is defined as \verb|N|, and everywhere \verb|N| appears in the code the literal value \verb|32| will be substituted before compilation.
\par
As a result of the need to store constant values in memory being removed, retrieval time from memory when a constant value is required has been eliminated. The literal value is immediately available. Other possible optimisations will be discussed in Section \ref{ss:fw} (Future Work).
\par
The overall goal of this project is to investigate whether this technique does provide any performance benefit, however any performance increase that incurs unacceptable deviation from the expected result is not a useful benefit. Therefore, the addition of defined constants should not reduce the accuracy of the result outside tolerance. It is not expected that it will.


\subsubsection{OP2 Library}
Outside the translation script, some OP2 API functions may need to be implemented differently in the OP2 library files, as the functions may require additional information to be stored and retrieved at runtime. It is a requirement that the OP2 API itself is not altered in any way by modifications to the library, so that all existing programs currently using the API will be able to seamlessly update to using the modified version.

\subsection{Testing \& Benchmarking}
Once the code generation stage has been completed, and the Python Script is able to generate valid C and CUDA code that can be compiled without error for an example OP2 Application, the resulting binary needs to be tested to ensure the result is correct, and its runtime needs to be benchmarked to determine if there is performance gain.
\par
The runtime will be compared against the same application generated for graphics cards without JIT compilation, to see if there is any benefit. Benchmarking results will include the time taken to recompile at runtime for the JIT compiled version, and the time taken to copy constant values to device memory for the AOT compiled version.


\clearpage
\subsection{New System Model}

\begin{figure}[h!]
\makebox[\textwidth][c]{
\resizebox{1.1\textwidth}{!}{
  \begin{tikzpicture}[node distance=4cm, auto]
    \node [file] (source) {Source Files};
    \node [block, below of=source] (op2py) {CodeGen script};
    \node [file, right of=op2py, text width=6em] (sourceOp) {Modified Application Files};
    \node [file, above of=sourceOp] (Kernels) {Kernels};
    \node [wblock, right of=sourceOp] (aot) {AOT Compiler};
    \node [wfile, below=1cm of aot] (op2lib) {OP2 Library Files};

    \node [rbin, right of=aot] (binary) {Program Binary};
    \node [rfile, above of=binary] (input) {Input Map};
    \node [rblock, right of=binary] (result) {Program Result};

    \node [file, below=4cm of sourceOp] (recKernels) {Optimised Kernels};
    \node [jfile, below=4cm of binary] (consts) {Constants};
    \node [jblock, below of=consts] (jit) {JIT Compiler};
    \node [jbin, right of= consts] (so) {Shared Object};

    \tikzset{dotted box1/.style={draw=black!100, dash pattern=on 4pt off 4pt,
      inner sep=8mm, rectangle, rounded corners, line width=2pt}};
    \tikzset{dotted box2/.style={draw=black!100, dash pattern=on 4pt off 4pt,
      inner sep=6mm, rectangle, rounded corners, line width=2pt}};

    \node (run-time) [dotted box1, fit = (input) (jit) (result), color=green!50!black] {};
    \node (jitbox) [dotted box2, fit = (recKernels) (so) (jit), color=red!50!black] {};

    \node at (run-time.north east) [above left=2mm] (rtbox) {\textbf{Run-time}};
    \node at (jitbox.south west) [below right=2mm] (corner) {\textbf{OP2 JIT Compilation}};

    \node (code-gen) [dotted box1, fit = (source) (recKernels), color=blue!50!black] {};
    \node at (code-gen.north west) [above right=2mm] (rtbox) {\textbf{Code Generation}};

    \path [line, color=blue!50!black] (source) -- (op2py);
    \path [line, color=red!70!black] (op2py) -- (recKernels);
    \path [line, color=blue!50!black] (op2py) -- (sourceOp);
    \path [line, color=blue!50!black] (op2py) -- (Kernels);
    \path [line, color=blue!50!black] (sourceOp) -- (aot);
    \path [line, color=blue!50!black] (Kernels) -- (aot);
    \path [line, color=black] (op2lib) -- (aot);
    \path [line, color=red!70!black] (op2lib) |- ($(jit.west)!0.2!(jit.north west)$);
    \path [line, color=green!50!black] (aot) -- (binary);
    \path [line, color=green!50!black] (input) -- (binary);
    \path [line, color=green!50!black] (binary) -- (consts);
    \path [line, color=red!70!black] (recKernels) |- ($(jit.west)!0.2!(jit.south west)$);
    \path [line, color=green!50!black] (consts) -- (jit);
    \path [line, color=green!50!black] (jit) -| (so);
    \path [line, color=green!50!black] (so) -- (binary);
    \path [line, color=green!50!black] (binary) -- (result);

  \end{tikzpicture}
  }}
  \caption{OP2 System Diagram with JIT Addition}
  \label{fig:jit_sys}
\end{figure}

\noindent Figure \ref{fig:jit_sys} describes the new workflow of the OP2 library, with Just-In-Time compilation (denoted with a dotted red box). As before, code generation takes an input of the application and optional additional files, and generates the Kernels and Modified Application Files. It also generates an additional set of Optimised Kernels, which contain code that will only be compiled inside the green box denoting `run-time', at which point the constants from the Input Map are known to the program. These Kernel files are not seen by the Ahead-Of-Time compiler.
\par
The JIT compilation will take place during the execution of the binary, and will therefore make up part of the program's execution duration. It will need to link the Optimised Kernels against the OP2 Library Files, so it is necessary that they are stored in a location that is also accessible at run-time, not just when the executable is compiled.
\par
The result of JIT compilation will be a Shared Object file, otherwise known as a Dynamically Loaded Library (DLL) file, with a standardised name. The program can then load this Shared Object and utilise its exported functions, which will be the recompiled versions of each parallel loop. As black boxes the two are equivalent (i.e.\ they have the same inputs and outputs), however theoretically the recompiled versions are faster to execute than the original versions.

\tinytitle{Ahead-Of-Time Kernels}

\noindent The Kernels compiled ahead of time could be altered such that their sole purpose is to invoke the compiler at runtime, then pass execution over to the JIT compiled function. It will be beneficial, however, to allow executables with the JIT compilation feature enabled or disabled to be compiled from the same source code.
\par
This requires the Ahead-Of-Time Kernels to retain the ability to execute the loop body without requiring JIT compilation, as well as being able to initiate the runtime compilation of the optimised kernels. Constant values should still be copied to device memory, but only if the re-compiled kernels will not be used, and copying the values is necessary.
\clearpage
\subsection{Library Modifications}
In the OP2 library, the main API function that will need to be modified is:
\codeline{void op_decl_const(int dim, char *type, T *dat, char *name)}{\cite[p9]{manual}}
\noindent This function is used to declare a constant value, its dimension, data type, and identifier.
Previously, this function copied the value to a device symbol, so that when required it could be read from device memory. In the implementation for this project, it will need to maintain a de-duplicated list of identifiers, and persist their values, data types, and dimensions. These parameters will be used when the first parallel loop is invoked to generate a header file, at which point it is known that no more constants can be declared. In the header file, each of the constant values will have a \verb|#define| directive making the value available as a literal value.\par
In the next section, the completed implementation is discussed. The contents of the files generated by the Python code generation script will be examined in more detail, and design decisions that were made will be explained. The Implementation is presented in its finished form, however the development process is covered later, in Section \ref{ss:pm}.
